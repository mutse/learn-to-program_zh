% chapter 2: Letters
\chapter{字符}

至此,我们已经学了几乎有关数字的全部知识,那么关于字母、单词和文字呢?

我们称程序中的字符集合为字符串(你可能会认为是标题中一连串的打印体字母)。为了便于看出源码中哪部分是字符串,我将它标成红色。下面是一些字符串:\\
\textcolor{red}{
\textquoteleft{Hello.}\textquoteright\\
\textquoteleft{Ruby rocks.}\textquoteright\\
\textquoteleft{5 is my favorite number... what is yours?}\textquoteright\\
\textquoteleft{Snoopy says \#\%\^{}?\&*@! when he stubs his toe.}\textquoteright\\
\textquoteleft{     }\textquoteright\\
\textquoteleft{}\textquoteright
}

综上所述,字符串由标点符号、数字、字母和空格组成,而不仅仅是字母。最后一个字符串中不包含任何内容,我们称之为空字符串。

我们习惯使用puts打印数字,不妨试一试使用它打印下面的一些字符串:
\begin{lstlisting}[language=ruby]
puts 'Hello, world!'
puts ''
puts 'Good-bye.'
\end{lstlisting}
\textcolor{red}{
Hello, world!\\
Good-bye.
}

问题解决得很棒。现在请您亲自尝试下一些字符串。

\section{字符串算术运算}

正如您可以做数值的算术运算,同样您也可以做字符串的算术运算。当然,您可以任意加字符串。
让我们将两个字符串相加,看看puts输出什么结果。
\begin{lstlisting}[language=ruby]
puts 'I like' + 'apple pie.'
\end{lstlisting}
\textcolor{red}{
I likeapple pie.
}

喔!我忘记在字符串`I like'和`apple pie'之间放置空格符。通常情况下,空格符无关紧要;但是在字符串中很重要(正如一句名言所说:计算机不会做你所想做的事情,而只会做那些你告诉它们如何做的事情。)。让我们再试试:
\begin{lstlisting}[language=ruby]
puts 'I like ' + 'apple pie.'
puts 'I like' + ' apple pie.'
\end{lstlisting}
\textcolor{red}{
I like apple pie.\\
I like apple pie.
}

(正如你所见,任何字符串加上空格字符都毫无影响。)

因此字符串可以相加,也可以相乘(无论如何,只能相乘数字)。请看:
\begin{lstlisting}[language=ruby]
puts 'blink ' * 4
\end{lstlisting}
撞击她的双眸\\
(开个小玩笑...... 其结果如下:)\\
\textcolor{red}{
blink blink blink blink
}

如果您思考下,就不难理解透彻。毕竟,7 * 3 意思是7 + 7 + 7,因此`moo * 3'表示`moo' + `moo' + `moo'。

\section{12 与 `12'}

在进一步阅读前,我们必须确保理解数字与进制数的区别。例如,12是数字,而`12'是二进制字符串。

不妨动手小试下牛刀:
\begin{lstlisting}[language=ruby]
puts  12  +  12
puts '12' + '12'
puts '12  +  12'
\end{lstlisting}
24\\
1212\\
12  +  12

试试这个如何:
\begin{lstlisting}[language=ruby]
puts  2  *  5
puts '2' *  5
puts '2  *  5'
\end{lstlisting}
10\\
22222\\
2  *  5

上述例子十分简明。尽管如此,如果您因不太细心而将字符串和数字混淆的话,您将遇到麻烦......

\section{问题}

在这个要点上您可能尝试过许多徒劳无功的试验。如果没有,这里有些:
\begin{lstlisting}[language=ruby]
puts '12' + 12
puts '2' * '5'
\end{lstlisting}
\#<TypeError: can't convert Fixnum into String>

嗯......这是一个错误的消息。问题在于您不能将一个数字和一个字符串相加,或将一个字符串和另一个字符串相乘。像下面这样做是毫无意义的:
\begin{lstlisting}[language=ruby]
puts 'Betty' + 12
puts 'Fred' * 'John'
\end{lstlisting}

需要注意的是:您可以在程序中写`pig'*5,因为它表达的意思是5组`pig'字符串相加在一起。然而,却不能写5*`pig',因为它表示`pig'组数字5,显然这是很荒谬的。

最后,如果想在程序中打印“You're swell!”,该怎么写? 可以试试下面的操作:
\begin{lstlisting}[language=ruby]
puts 'You're swell!'
\end{lstlisting}

是的,即便不运行,它也不起作用。因为计算机认为我们已处理过字符串(这就是使用具有语法颜色标识功能的文本编辑器的好处。)。如何才能让计算机知道字符串中的停顿?因此我们必须使用 \ 转义,如下:
\begin{lstlisting}[language=ruby]
puts 'You\'re swell!'
\end{lstlisting}
You're swell!

斜划线是一个转义字符。换而言之,若将斜划线和另一个字符放在一起,将转换成一个新的字符。
\begin{lstlisting}[language=ruby]
puts 'You\'re swell!'
puts 'backslash at the end of a string:  \\'
puts 'up\\down'
puts 'up\down'
\end{lstlisting}
You're swell!\\
backslash at the end of a string:  \textbackslash \\
up\textbackslash{down} \\
up$\backslash$down

因为斜划线并未转义`d'字符,而是转义了它本身,所以最后两个字符串是相同的。尽管它们在代码中是不相同的,但是计算机却认为它们是一样的。

若您还有其它的疑问,请继续阅读!毕竟,我不可能将本页中每一个问题作一一解答。
