% chapter 0
\chapter{初步}

当使用计算机编程时,您必须“说”计算机能理解的语言,也就是程序设计语言。然而世界上有众多不同的程序语言,其中许多语言非常出色。在本指南中,我挑选了自己最喜爱的程序设计语言\pozhehao{Ruby}。

除了是我最喜欢的程序设计语言,Ruby也是我见过(我见识过许多程序设计语言)最简易的程序设计语言。事实上,这也是我写这本指南的真实原因。之前我没打算写一本指南,然后选择了Ruby语言因为它是我的最爱。相反,我发现Ruby的简易性而决定应该为初学者写一份指南。这正是Ruby的简易性促使这本指南的诞生,而不是我的爱好。(使用其它程序设计语言编写一本简易的指南,比如C++或Java,将会花费大量的纸张。)但是不要因为Ruby的简易性而认为它是一种初学者编程语言。Ruby是一种十分强大且满足专业需求的程序设计语言。

当你使用人类语言书写东西时,你所书写的称之为文字。而当你使用计算机语言编写东西时,你所写的称之为代码。我已经包含大量的Ruby代码穿插本指南中,你可以在自己的电脑上运行其中大多数程序。为了便于简易阅读代码,我已将代码着上不同的颜色。(例如,数字总是绿色的。)假设你输入的内容将会以白盒子呈现,打印输出的内容将会以蓝盒子显示。

如果你遇到一些难以理解的问题,或是你有疑问而没人回答时,请记下问题继续阅读!很有可能问题的答案将会在后续章节中出现。如果在阅读完最后一章,你的问题还没得到解答,我会告诉你该去哪里询问。那里有许多极好的人,非常乐意帮你解答;你仅仅需要知道他们在哪里。

首先,我们需要下载和在你的电脑上安装Ruby。

\section{Windows 安装}

在Windows下安装Ruby是件轻而易举的事情。首先,你需要下载Ruby安装器\footnote{\href{http://rubyinstaller.rubyforge.org/}{Ruby installer}}。下载页面中可能有几个版本供你选择;这本指南使用的1.8.4版本,因此必须确保你下载的至少是最近版本。(我将尽可能地获取最新版本)然后简单地运行安装程序。它会询问你想将Ruby安装在什么地方。我通常采用默认位置安装,除非你有好的理由。

为了编程,你需要能写程序和运行它们。为了做到这,你将会需要一个文本编辑器和命令行。

Ruby安装器伴随着一个称作SciTE(Scintilla文本编辑器)的可爱编辑器。您可以从开始菜单中选中SciTE来运行。如果您想给您的代码涂上类似指南中的例子的颜色,请下载这些文件并将它们放在您的SciTE安装目录(如果选择默认目录,c:/ruby/scite)中:
\begin{itemize}
\item 全局属性
\item Ruby属性
\end{itemize}

创建一个目录用于保存你的程序是个不错的主意。当保存程序时,确保你将程序保存到了这个目录中。

开始使用命令行,选中开始菜单中附件文件夹下命令提示符。你需要定位到用来保存程序的目录中。输入cd ..切换到上级目录,cd foldername将进入名为foldername的目录。为查看当前目录中的所有文件,请输入dir /。

\section{Macintosh 安装}

如果你使用的是 Mac OS X 10.2 (Jaguar),然后准备在你的系统中安装Ruby。安装过程简单吗?很不幸地,我不认为你能在Mac OS X 10.1 以及更早的版本上使用Ruby。

为了方便编程,你需要能编辑和运行程序。为了做到这个,你需要一个文本编辑器和命令行终端。

你可以通过终端程序(在应用程序/工具中可以找到)使用命令行。

你可以使用任意一个你熟悉或舒服的文本编辑器。如果你使用TextEdit,确保保存程序为文本格式;否则程序无效。

\section{Linux 安装}

首先,如果你的电脑上已安装了Ruby,你需要核实和查看下。输入which ruby,如果显示信息如“/usr/bin/which: no ruby in ...”,你需要下载Ruby;否则使用ruby –v查看你所使用的Ruby版本。如果它比下载页面上最新的稳定包还旧,你需要升级。

如果你是root(超级管理员)用户,你可能不需要Ruby的任何安装命令。如果你不是,你可以告诉你的系统管理员安装它。(使用这种方式,系统上的每一个用户都可以使用Ruby。)

否则,你可以安装它,以至于仅仅只要你可以使用它。将下载的文件移至一个临时目录下,比如\$HOME/tmp。如果下载的文件名是ruby-1.6.7.tar.gz,你可以使用tar zxvf ruby-1.6.7.tar.gz打开它。从当前目录下切换到你刚才创建的目录下(在本例中,cd ruby-1.6.7)。

输入./configure –prefix=\$HOME配置你的安装。接着输入用来创建Ruby解释器的make命令,这可能会花上片刻时间。当完成后,输入make install命令来安装Ruby。

接着,通过编辑\$HOME/.bashrc文件,将\$HOME/bin添加到命令查找目录中。(你可能必须退出系统再登陆以确保生效。)当做完这,使用ruby -v测试安装。如果显示的是你安装的Ruby版本,现在你可以删除\$/HOOME/tmp目录中的所有文件(用来放置下载文件的目录)。

这就是安装过程。你可以准备看《学习程序设计》的下一章。

