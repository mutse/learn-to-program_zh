% chapter 4: Mixing It Up
\chapter{结合在一起}

我们已研究一些不同类型的对象,如数值和字符,并使用变量指向这些对象。接下来,我们想将它们组合在一起,漂亮地运行。

如你所见,如果想要一个程序打印25,下述代码将丝毫不起作用,因为不能将数字和字符串作相加运算:

\begin{lstlisting}[language=ruby]
var1 = 2
var2 = '5'

puts var1 + var2
\end{lstlisting}

问题的局限在于计算机不理解你是否要得到结果7(2+5),或是想得到字符串25(`2' + `5')。

在相加之前,必须使用某种方法获取变量var1的字符串形式,或是变量var2的整型形式。

\section{转换}

为了获取一个对象的字符串形式,只需在其后简单地写上 .to\_s:

\begin{lstlisting}[language=ruby]
var1 = 2
var2 = '5'

puts var1.to_s + var2
\end{lstlisting}
\textcolor{red}{
25
}

同样,使用to\_i 获取一个对象的整型形式,使用to\_f 获取浮点形式。让我们观察这三种方法在可以做或不能做的事情上多了些相近:
\begin{lstlisting}[language=ruby]
var1 = 2
var2 = '5'

puts var1.to_s + var2
puts var1 + var2.to_i
\end{lstlisting}
\textcolor{red}{
25\\
7
}

注意,即使调用to\_s方法获取变量var1的字符串形式,但变量var1仍然指向2,而不是字符`2'。除非明确地给变量var1赋值(需要用=赋值),否则在程序生命期内它将指向2。

现在,让我们亲自试一试下面一些非常有趣的(有点奇怪的)转换:

\begin{lstlisting}[language=ruby]
puts '15'.to_f
puts '99.999'.to_f
puts '99.999'.to_i
puts ''
puts '5 is my favorite number!'.to_i
puts 'Who asked you about 5 or whatever?'.to_i
puts 'Your momma did.'.to_f
puts ''
puts 'stringy'.to_s
puts 3.to_i
\end{lstlisting}
\textcolor{red}{
15.0 \\
99.999 \\
99 \\
\\
5 \\
0 \\
0.0 \\
\\
stringy \\
3
}

因此,这可能让你觉得吃惊。第一个表达式是相当标准的,其结果是15.0。接着,将字符串`99.999'分别转换成浮点数和整型。浮点数转换是我们预料的结果,而整型转换是四舍五入的结果。

接下来,将几个不同寻常的字符串转换成数值。to\_i方法会忽略无法识别的字符,而指向剩余的字符串。因此,第一个表达式转换结果是5;而其它的表达式,因为以字母开头,全部被忽略,所以计算结果选取0。

最后,最后两个表达式没做任何转换,与我们预期的结果一致。

\section{窥探 puts 方法}

我们常用的这个方法有些古怪...看看下面的例子:

\begin{lstlisting}[language=ruby]
puts 20
puts 20.to_s
puts '20'
\end{lstlisting}
\textcolor{red}{
20\\
20\\
20
}

为何三个表达式会输出相同的结果?最后两个相同,是因为20.to\_s的结果是`20'。而第一个,整型20?写下整数20意味着什么?正因为如此,当在一张纸上写下2和0时,你所写下的是一个字符串,并非一个整数。整数20是手指头和脚指头的个数,而不是2后面紧跟着0。

我们的朋友puts背后隐藏着一个大秘密:在使用puts试图写一个对象之前,它会使用to\_s方法获取该对象的字符串形式。事实上,puts中的s表示字符串,puts实际意思是输出字符串。

或许这并不那么激动万分,Ruby中有许多的对象(您将学习如何掌握这些方法)。如果尝试使用puts一个奇怪的对象,比如祖母的照片、一首音乐或一些别的东西,会产生什么结果。这些将在后文出现...

在此期间,我们已经给您介绍了一些方法,这足以让我们编写一些有趣的程序...

\section{gets和chomp方法}

如果puts方法意味着输出字符串,我确信你能猜出gets代表什么意思。正如同puts方法用于输出字符串,gets方法仅仅用于取回字符串。gets方法将从何处取回字符串?

当然,是从键盘输入中获取。因为只有键盘才能产生字符串,就完美地解决这个问题。若将gets方法直接放在那里,读取你所输入的内容至按下Enter键为止,会发生什么?请尝试下例:

\begin{lstlisting}[language=ruby]
puts gets
\end{lstlisting}
\textcolor{red}{
Is there an echo in here? \\
Is there an echo in here?
}

当然,你所输入的任何内容将重复回显。请尝试数次输入不同的内容。

至此,我们可以编写交互的程序!在本例中,输入你的姓名,它将会问候你:
\begin{lstlisting}[language=ruby]
puts 'Hello there, and what\'s your name?'
name = gets
puts 'Your name is ' + name + '?  What a lovely name!'
puts 'Pleased to meet you, ' + name + '.  :)'
\end{lstlisting}

呃!运行程序--我输入自己的名字,输出结果如下:
\textcolor{red}{
Hello there, and what's your name? \\
Chris \\
Your name is Chris \\
?  What a lovely name! \\
Pleased to meet you, Chris \\
.  :)
}

嗯...看起来像是,当我输入字符C、h、r、i、s,然后按下Enter键,gets方法获取名字中所有的字符和Enter。很幸运的是,有一个方法仅用来排序的:chomp。它用来删除Enter并分配字符串结束符。请再次尝试此程序,这次chomp方法可以帮助我们:
\begin{lstlisting}[language=ruby]
puts 'Hello there, and what\'s your name?'
name = gets.chomp
puts 'Your name is ' + name + '?  What a lovely name!'
puts 'Pleased to meet you, ' + name + '.  :)'
\end{lstlisting}
\textcolor{red}{
Hello there, and what's your name? \\
Chris \\
Your name is Chris?  What a lovely name! \\
Pleased to meet you, Chris.  :)
}

很棒!请注意,因为name变量指向gets.chomp方法,但我们并不这样说name.chomp。

\section{尝试一些事情}

\begin{itemize}
\item 编写一个程序,用于询问一个人的名字、中间名和姓氏。最后,需要使用全名问候这个人。
\item 编写一个程序,用于询问一个人喜欢的数字。程序能将这个数字加1,建议结果是个很大的、中意的数字。(务必得体)
\end{itemize}

一旦你完成这两个程序(你愿意尝试其它也行),让我们开始学习关于更多的方法。

