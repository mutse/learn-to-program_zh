% chapter 3: Variables and Assignment
\chapter{变量与赋值}

So far, whenever we have putsed a string or a number, the thing we putsed is gone. What I mean is, if we wanted to print something out twice, we would have to type it in twice: \\

\begin{lstlisting}[language=ruby]
puts '...you can say that again...'
puts '...you can say that again...'
\end{lstlisting}
\textcolor{red}{
...you can say that again...\\
...you can say that again...
}

It would be nice if we could just type it in once and then hang on to it... store it somewhere. Well, we can, of course—otherwise, I wouldn't have brought it up! 

为了将字符串保存在计算机内存中,必须给字符串命名。通常,程序员称之为赋值,然后调用这些命名的变量。变量可以是任意字母和数字组成的序列,但首字符必须是小写字母。让我们再试一试上述最后一个程序,但这次将字符串命名为myString(当然也可将之命名为str或myOwnLittleString或henryTheEighth)。\\
\begin{lstlisting}[language=ruby]
myString = '...you can say that again...'
puts myString
puts myString
\end{lstlisting}
\textcolor{red}{
...you can say that again...\\
...you can say that again...
}

不论何时你厌烦使用myString时,程序都会用字符串`...you can say that again...'替换之。可以将变量myString认为是“指向”字符串`..you can say that again...'。下面有一个小巧有趣的例子:

\begin{lstlisting}[language=ruby]
name = 'Patricia Rosanna Jessica Mildred Oppenheimer'
puts 'My name is ' + name + '.'
puts 'Wow!  ' + name + ' is a really long name!
\end{lstlisting}
\textcolor{red}{
My name is Patricia Rosanna Jessica Mildred Oppenheimer.\\
Wow!  Patricia Rosanna Jessica Mildred Oppenheimer is a really long name!
}

不仅可以给变量赋值一个对象,也可以同一个变量重赋值不同的对象(这就是我们称之为变量的原因:因为她们所指向的内容可改变。)。
\begin{lstlisting}[language=ruby]
composer = 'Mozart'
puts composer + ' was "da bomb", in his day.'

composer = 'Beethoven'
puts 'But I prefer ' + composer + ', personally.'
\end{lstlisting}
\textcolor{red}{
Mozart was ``da bomb", in his day.\\
But I prefer Beethoven, personally.
}

当然,变量可以指向任何类型的对象,而不仅仅是字符串:
\begin{lstlisting}[language=ruby]
var = 'just another ' + 'string'
puts var

var = 5 * (1+2)
puts var
\end{lstlisting}
\textcolor{red}{
just another string\\
15
}

事实上,变量可以指向任何类型,除了其它变量外。如果尝试如下操作,其结果会是?
\begin{lstlisting}[language=ruby]
var1 = 8
var2 = var1
puts var1
puts var2

puts ''

var1 = 'eight'
puts var1
puts var2
\end{lstlisting}
8\\
8\\
\\
eight\\
8\\

首先,当将变量var2指向变量var1,其结果指向8(就如同变量var1所指向的值)。然后,将变量var1指向字符串`eight',而变量var2并未指向变量var1,因此它仍指向8。

至此,我们已学习了变量、数字和字符串,下一章学习结合在一起。
